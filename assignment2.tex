\documentclass[journal,12pt,twocolumn]{IEEEtran}

\usepackage{setspace}
\usepackage{amssymb}
\usepackage{amsthm}
\usepackage{mathrsfs}
\usepackage{longtable}
\usepackage{enumitem}
\usepackage{mathtools}
\usepackage{longtable}
\usepackage{booktabs}
\usepackage[breaklinks=true]{hyperref}

\usepackage{listings}
    \usepackage{color}                                            %%
    \usepackage{array}                                            %%
    \usepackage{longtable}                                        %%
    \usepackage{calc}                                             %%
    \usepackage{multirow}                                         %%
    \usepackage{hhline}                                           %%
    \usepackage{ifthen}                                           %%
    \usepackage{lscape}     
    \usepackage{amsmath}
      
\def\inputGnumericTable{}

\bibliographystyle{IEEEtran}
\providecommand{\pr}[1]{\ensuremath{\Pr\left(#1\right)}}
\providecommand{\brak}[1]{\ensuremath{\left(#1\right)}}

\parindent 0px

\begin{document}
\vspace{3cm}
\title{ASSIGNMENT 2}
\author{BT21BTECH11005 - MANIKANTA}

\maketitle
\textbf{PROBLEM}:-Given three identical Boxes $A$, $B$ and $C$, Box $A$ contains 2 gold and 1 silver coin, Box $B $contains 1 gold and 2 silver coins and Box $C$ contains 3 silver coins. A person chooses a Box at random and takes out a coin. If the coin drawn is of silver, find the probability that it has been drawn from the Box which has the remaining two coins also of silver.\\

\textbf{SOLUTION}:-\\
given,three boxes are equally likely\\
so,\\
\begin{enumerate}[label=]
\item $\displaystyle P(A)= \dfrac{1}{3},\displaystyle P(B)= \dfrac{1}{3},\displaystyle P(C)= \dfrac{1}{3}$
\end{enumerate}

NO. OF GOLD COINS (GC) and SILVER COINS(SC) in the given three boxes are:-
\begin{table}[ht!]
 \begin{tabular}{|c|c|}
\hline
 \textbf{EVENT} & \textbf{DESCRIPTION}\\
\hline
 $X = 0$ & coin drawn from box $A$ \\
\hline
 $X = 1$ & coin drawn from box $B$\\
\hline
 $X = 2$ & coin drawn from box $C$\\
\hline
 $Y = 0$ & coin drawn is SILVER\\
 \hline
 $Y = 1$ & coin drawn is GOLD\\
 \hline
\end{tabular}\\

\caption{NO. OF COINS IN THREE BOXES}
\label{Tables : TABLE}
\end{table}\\

Probability of GOLD COINS(GC) snd SILVER COINS(SC) in the given three boxes are:-
 \begin{table}[ht!]
 \begin{tabular}{|c|c|}
\hline
 \textbf{PROBABILITY} & \textbf{VALUE}\\
\hline
  \pr{ X = 0} & $\displaystyle  \dfrac{1}{3}$ \\
\hline
  \pr{ X = 1} & $\displaystyle  \dfrac{1}{3}$ \\
\hline
 \pr{ X = 2} & $\displaystyle  \dfrac{1}{3}$ \\
\hline
 \pr{ Y = 0 \mid X = 0} & $\displaystyle  \dfrac{1}{3}$ \\
\hline
 \pr{ Y = 0 \mid X = 1 } & $\displaystyle  \dfrac{2}{3}$ \\
\hline
 \pr{ Y = 0 \mid X = 2} & $1$ \\
\hline
 \pr{ X = 2 \mid Y = 0} & $???$ \\
\hline
\end{tabular}\\

\caption{probability of GC and SC in three boxes}
\label{Tables : TABLE}
\end{table}\\

now,
by using BAYES THEOREM\\
Probability that the coin drawn is silver from box $C$,is equal to
\begin{align}
P(C \mid S) &= \frac{P(C)P(\frac{S}{C})}{P(A)P(\frac{S}{A})+P(B)P(\frac{S}{B})+P(C)P(\frac{S}{C})}\\
 &= \frac{\frac{1}{3} \times 1}{\frac{1}{3} \times \frac{1}{3}+\frac{1}{3} \times \frac{2}{3}+\frac{1}{3} \times 1
 }\\
 &= \frac{\frac{1}{3}}{\frac{1}{9}+\frac{2}{9}+\frac{1}{3}}\\
 &= \frac{1}{3} \times \frac{9}{6}\\
 P(C \mid S) &= \frac{1}{2}
\end{align}
$\therefore$ from$(5)$,\\
\begin{enumerate}[label=]
\item Probability that the coin drawn is silver from box $C$ = $\frac{1}{2}$.
\end{enumerate}
\end{document}
